\documentclass[12pt]{article}
\usepackage[utf8]{inputenc}
\usepackage{amsmath}
\usepackage{amssymb}
\usepackage{enumitem}
\usepackage{mathtools}
\usepackage{algorithmic}
\usepackage{hyperref}
\DeclarePairedDelimiter\ceil{\lceil}{\rceil}
\DeclarePairedDelimiter\floor{\lfloor}{\rfloor}
\usepackage{amssymb}
\newcommand*{\field}[1]{\mathbb{#1}}
\begin{document}
\title{\vspace{-5cm} Theoretical Computer Science Coursework}
\author{Jake Mortimer}
\maketitle
\begin{enumerate}
\section{Complexity and Approximability} 
\item \textbf{Question 1}
\begin{enumerate}
\item 
i. Prove that SHORT PATH $\in \textbf{NL}$ \\
Given an instance of the SHORT PATH we need to show that there is a non-deterministic Turing Machine that decides it in log-space. Starting at the start vertex $s$ of G we have a counter $c$ in memory. At each vertex in the path from $s$ we non-deterministically guess the next vertex. If we reach $t$ and the value in the counter $c \leq k$ then accept. If $c>k$ then reject. This obviously solves SHORT PATH in logarithmic space because the counter can have a maximum value of $|V|$. If the counter were greater than $|V|$ then this would mean that we have visited one vertex of the input graph more than once and, hence, there is a cycle. $\therefore $ We could reduce the path length to some path length $\leq |V|$ by simply removing the cycle, if such a path existed from $s$ to $t$ $\Rightarrow c$ can be stored in logarithmic space and SHORT PATH $\in \textbf{NL}$.\\
ii. Describe the reduction REACHABILITY $\leq _{L} $ SHORT PATH \\
We are initially given an instance of REACHABILITY; this is a graph $G = (V,E)$ and two vertices $s,t$. A yes instance of REACHABILITY is where $t$ is reachable from $s$ by a directed path on the edges of $G$. To create an instance of SHORT PATH in logarithmic space we use the following reduction:\\
The instance of SHORT PATH consists of the graph $G'=(V',E')$, the vertices $s',t' \in V'$ between which we find a path of length at most $k=|V|+1$. For each vertex $v \in V$ we create the vertices $v^1,v^2,...,v^{|V|}$ and add these to $V'$. Also, create the vertices $s',t'$ and add them to $V'$. For each directed edge $(u,v) \in E$ we create the undirected edges $(u^i,v^{i+1})$ for all $i \leq |V|-1$ and add these to $E'$. Also, create the edges $(s',s^1),(t^1,t^2),(t^2,t^3),...,(t^{|V|},t')$ and add these to $E'$. Finally, to complete the reduction we set $k=|V|+1$. This can obviously be done in logarithmic space, therefore, the reduction conforms with space requirements.\\
iii. Prove the correctness of the above reduction\\
In order to prove the correctness of the above reduction I will prove that a yes instance of REACHABILITY $\Rightarrow $ yes instance of SHORT PATH, and a no instance of REACHABILITY $\Rightarrow $ no instance of SHORT PATH. This is sufficient to show that SHORT PATH is $\textbf{NL}$-hard, and, when combined with the proof of membership of $\textbf{NL}$ above, proves that SHORT PATH is $\textbf{NL}$-complete.\\
\textbf{Note}: $v^i$ means the vertex $v$ with \textit{row} $i$

First we prove that a yes instance of REACHABILITY $\Rightarrow $ yes instance of SHORT PATH:\\
Suppose that we have a yes instance of REACHABILITY. This would mean that $t$ is reachable from $s$ in the directed graph $G$ (defined in the instance of REACHABILITY). Let us construct an instance of SHORT PATH using the above reduction. This would yield the instance $G'=(V',E')$, $k=|V|+1$, and the start and end vertices $s',t'$. If $t$ is reachable from $s$ in $G$ then this means that there is some path of directed edges starting at $s$ and ending at $t$. This directed path $P=(s,u),(u,v),...,(w,t)$ for some $u,v,w \in V$ would become the following path in $G'$ $P'=(s',s^1),(s^1,u^2),(u^2,v^3),...,(w^{l-1},t^l),(t^l,t^{l+1}),...,(t^{k-1},t^k),(t^k,t')$. As can be seen the superscript, which corresponds to the rows of the vertices, monotonically increases by 1 each time if the edges traversed in $G'$ are traversed in the correct direction as defined by the digraph $G$. If some row were visited more than once this would mean that one of the edges corresponding to the directed edge in $G$ would have been traversed in the wrong direction. For the constructed instance to be a yes instance $|P'|\leq k =|V|+1$. If the latter of the above cases were true and we visited one row multiple times then this would add at least 2 to the length of $P'$. There are $|V|+1$ rows in the graph excluding the start vertex; if we were to traverse an edge in the wrong direction we would get $|P'| \geq |V|+3 > |V|+1 \Rightarrow$ this would be a no instance of SHORT PATH. $\therefore$ we can only have a yes instance if all the edges of $G'$ are traversed so the row value increases monotonically, which corresponds to all the edges being traversed in the correct direction in $G$. \\
Secondly, we need to prove that a no instance of REACHABILITY $\Rightarrow$ no instance of SHORT PATH: \\
Suppose that we have a no instance of REACHABILITY. This would mean that $t$ is not reachable from $s$, which in turn means that there is no path from $s$ to $t$ in $G$. Let us construct the corresponding instance of SHORT PATH using the above reduction. This would yield the instance $G'=(V',E')$, $k=|V|+1$, and the start and end vertices $s',t'$. \\
Assume, for a contradiction, that there is a path in $G'$ from $s'$ to $t'$ of length at most $k=|V|+1$, which we call $P'$. This would correspond to a yes instance of SHORT PATH. As such, using the logic above each row can only be visited once otherwise $|P'|>k=|V|+1$, which would be a no instance of SHORT PATH. We reverse the reduction from SHORT PATH to REACHABILITY to get $G,s,t$, and transform the path $P'$ to $P$ on $G$, in the following way $P'=(s',s^1),(s^1,u^2),(u^2,v^3),...,(w^{l-1},t^l),(t^l,t^{l+1}),...,(t^{k-1},t^k),(t^k,t')$ would become This directed path $P=(s,u),(u,v),...,(w,t)$ for some $u,v,w \in V$. This would trivially yield a path from $s$ to $t$ in $G$, and hence we would have a yes instance of REACHABILITY. This is a contradiction $\Rightarrow$  no path from $s'$ to $t'$ in at most $k$ steps in the graph $G'$. $\therefore$ the constructed instance of SHORT PATH is a no instance.\\
This means YES $\Rightarrow$ YES and NO $\Rightarrow$ NO, and so we have correspondence of yes instances of these problems. \\$\therefore$ SHORT PATH is $\textbf{NL}$-complete $\square $

\item Show that SHORT-PATH-2019 $\in \textbf{L}$ \\
The following proof that SHORT-PATH-2019 $\in  \textbf{L} $ will be constructive, as such I will present an algorithm to solve the problem that works in logarithmic space.\\
I will first introduce a useful syntactic structure:\\
$\textbf{for}$ $v_1,v_2,...v_l \in V$ $\equiv $ $\textbf{for}$ $v_1 \in V \{ \textbf{for}$ $v_2 \in V \{ ... \{\textbf{for}$ $v_l \in V \{...\}\}\}\} $\\
The algorithm is as follows:
\begin{algorithmic}
\FOR {$v_1,v_2 \in V$}
	\IF {sequence of vertices is valid path from $s\rightarrow t $}
		\STATE Accept
	\ENDIF
\ENDFOR
\FOR {$v_1,v_2,v_3 \in V$}
	\IF {sequence of vertices is valid path from $s\rightarrow t $}
		\STATE Accept
	\ENDIF
\ENDFOR\\
...
\FOR {$v_1,v_2,v_3,...,v_{2017} \in V$}
	\IF {sequence of vertices is valid path from $s\rightarrow t $}
		\STATE Accept
	\ENDIF
\ENDFOR
\FOR {$v_1,v_2,v_3,...,v_{2017},v_{2018} \in V$}
	\IF {sequence of vertices is valid path from $s\rightarrow t $}
		\STATE Accept
	\ENDIF
\ENDFOR
\STATE Reject
\end{algorithmic}
The sequence of vertices that are checked when determining the validity of a path are $(s,v_1,...,v_l,t)$. These paths are of length at most 2019 $\Rightarrow $ the space used to store each enumerated path is bounded by a constant $\Rightarrow $ the construction trivially uses logarithmic space and solves the problem SHORT-PATH-2019.
$\therefore $ SHORT-PATH-2019 $\in \textbf{L}$ $\square$

\end{enumerate}
\item \textbf{Question 2}
\begin{enumerate}
\item Prove unconditionally that TWO CLIQUES is not \textbf{PSPACE}-complete with respect to log space reductions.\par
We can show that TWO CLIQUES problem is a member of $\textbf{L} $. This can be shown by the proving the reduction TWO CLIQUES $\leq _{L}$ BIPARTITE which is a member of $\textbf{L} $. BIPARTITE is defined as follows:\\
\textit{Instance:} Graph $G=(V,E)$\\
\textit{Question:} Is the graph G bipartite?\\
The reduction is simple; given an instance of TWO CLIQUES (an undirected graph $G$) we take the complement of the graph, resulting in graph $\bar{G}$ which we give as the instance of BIPARTITE.\\
\textit{\textbf{Proof that BIPARTITE $\in \textbf{L} $}}\\
We will be using Reingold's Theorem that $s\rightarrow t$ connectivity in an undirected graph is in $\textbf{L} $, \href{https://en.wikipedia.org/wiki/SL_(complexity)}{seen here - \textit{SL and Reingold}}. With the knowledge that UPATH (does there exist a path from $s\rightarrow t$ in $G$?), we prove that BIPARTITE $\in \textbf{L} $ by reducing to co-UPATH (The complement of UPATH; there is no path from $s\rightarrow t$ in $G'$), as $\textbf{L} = co-\textbf{L} $ trivially. \\
For a given bipartite graph $G=(V,E)$ we can define the graph $G'$ such that $V'=\{v_1,v_2\mid v \in V\} \cup \{s,t\} $ and $E'=\{(u_1,w_2),(u_2,w_1) \mid (u,w)\in E\} \cup \{(s,v_1) \mid \forall v \in V\} \cup \{(v_2,t) \mid \forall v \in V\}$.
This reduction can clearly be done in logarithmic space. Now we have a reduction we need to prove the correspondence of yes-instances: \\
($\Rightarrow $) Given a yes instance of BIPARTITE, a graph $G$, we can use the reduction above to get the graph $G'$. We substitute this graph into the co-UPATH problem. We pass in the vertices $s,t$, and if it accepts; there are no paths between any of the expanded vertices. This is the case because for there to be a path between $s$ and $t$ there would need to be a path between two vertices such as $v_1,v_2 \in V'$ there must be an edge within one of the sets of vertices of $G$. \\Assume, for a contradiction, that there was not an edge between any two vertices of the same set in $G$ but there was a path $v_1\rightarrow v_2$. The same disjoint sets in $G$ can be applied to $G'$, and we have established that there are no edges within a set in $G$, similarly, this is a fact for these sets in $G'$. This means that $v_1,v_2$ would have to be connected by some vertex in the opposite set $(v_1,d_1),(v_2,d_1)$ for there to be a path between them. This is not possible: it would mean that in the original graph $G$ there would have been the edges $(v,d),(d,v)\in E$, and in an undirected graph there would only be one of these edges. Therefore, we have a contradiction meaning that for there to be a path $v_1\rightarrow v_2$ there would have to be an edge within the bipartite sets in $G$, and in turn a yes instance of BIPARTITE $\Rightarrow $ yes instance of co-UPATH. \\
($\Leftarrow $) Given a yes instance of co-UPATH which corresponds to the above reduction we transform $G'$ to $G$. For the instance to be a yes instance, there can be no path $s\rightarrow t$ which means no pair of vertices $v_1,v_2 \in V'$ can have a path between them, otherwise there would be a path from $s\rightarrow t$. When we reduce $G'$ to $G$ we first remove $s,t$, this leaves two sets of disjoint vertices with no edges between vertices of the same set. Assume, for a contradiction, that the two sets had an edge within one of the sets, this would mean that $(v_1,w_2)$ would be an edge for some $v_1,w_2 \in V'$ which are in the same set. From the construction, this would mean that there is a path from $s\rightarrow t$. $\therefore $ we have a contradiction $\Rightarrow $ there can be no edges within these disjoint sets. We then construct $G=(V,E) $ from the pairs of vertices $v_1,v_2 \in V'$, and the edges $(v_1,w_2),(v_2,w_1) \in E'$ as described in the reduction. This results in a graph $G$ which, trivially, is bipartite: two disjoint sets maintained from $G'$ and there can be no edges between two vertices of a set. $\therefore $ yes instance of co-UPATH $\Rightarrow $ yes instance of BIPARTITE. \\
$\therefore $ BIPARTITE $\leq_{L} $ co-UPATH $\leq_{L} $ UPATH $\Rightarrow $ BIPARTITE $\in \textbf{L} $ $\square $\\
\textit{\textbf{Continuing from before proof...}}\\
We now prove the reduction TWO CLIQUES $\leq _{L}$ BIPARTITE. \\ 
\textit{\textbf{Proof of Reduction}}\\  
To prove the correctness of this reduction we need to prove the correspondence of yes instances. This means that we have a yes instance of TWO CLIQUES $\iff $ we have a yes instance of BIPARTITE.\\
($\Rightarrow $) Given a yes instance of TWO CLIQUES $G, V_1, V_2$ where $G$ is the graph, and $V_1,V_2$ are the two cliques that the graph can be subdivided into. We take the complement of $G$, $\bar{G}$ as in the reduction. In the initial yes-instance for TWO CLIQUES, $V_1$ and $V_2$ are  completely connected sub-graphs; each vertex of either is connected to every other vertex of the same clique. When we take the complement of $G$, the edge set contains the edges which were not present in the initial graph. This means that $V_1,V_2$ will have no edges internally in $\bar{G}$ and only edges between vertices of opposite sets. As such the graph $\bar{G}$ is bipartite. $\therefore $ yes instance of TWO CLIQUES $\Rightarrow $  yes instance of BIPARTITE.
\\
($\Leftarrow $)Given a yes instance of BIPARTITE $\bar{G}, V_1, V_2 $ we take the complement of the graph $\bar{G} $ and get $G$. By the reverse of the above argument it is clear that a yes instance of BIPARTITE $\Rightarrow $ yes instance of TWO CLIQUES.
\\
$\therefore $ TWO CLIQUES $\in \textbf{L} $ $\square $ \\
\textit{\textbf{Continuing from before proof...}}\\  
We now have that TWO CLIQUES $\in \textbf{L}$. It is known that $\textbf{L} \subseteq \textbf{NL} \subseteq \textbf{PSPACE} $ because if a problem is solvable in logarithmic space on a deterministic machine it is obviously solvable in logarithmic space on a \textit{non}-deterministic machine $\Rightarrow $ $\textbf{L} \subseteq \textbf{NL} $. It is also clear to see that if a problem is solvable in logarithmic space on a non-deterministic machine then it is solvable in polynomial space on a deterministic machine by Savitch's Theorem $\Rightarrow$  $\textbf{NL} \subseteq \textbf{PSPACE} $. \\
\\
If $\textbf{L} \subseteq \textbf{NL}$ and $\textbf{NL} \subseteq \textbf{PSPACE}$, then the above sequence of set containments holds true.\\
We now attempt to derive a contradiction: assume that TWO CLIQUES is \textbf{PSPACE}-complete. This would mean that $\textbf{L} = \textbf{PSPACE} \Rightarrow \textbf{PSPACE} \subseteq \textbf{NL}$. We already have that $\textbf{NL} \subseteq \textbf{PSPACE}$, so combining these we get that $\textbf{NL} = \textbf{PSPACE} $ which we can prove not to be true.\\
\textit{\textbf{Proof of $\textbf{NL} \neq \textbf{PSPACE}$}}\\
First, we need to define space constructible functions. A function $f:\field{N} \rightarrow \field{N} $ such that $f(n) \geq log(n)$ is called space constructible if the function that maps $1^n$ to the binary representation of $f(n)$ is computable in space $O(f(n))$ From this it is clear that if $f(n)$ and $g(n)$ are space constructible functions, and $f(n)=o(g(n))$ then $SPACE[f] \subset SPACE[g]$. By Savitch's Theorem, $\textbf{NL} \subseteq SPACE[log^2(n)]$. Since $log^2(n) = o(n)$, we have $SPACE[log^2(n)] \subset SPACE[n]$ $\therefore \textbf{NL} \neq \textbf{PSPACE}$ $\square $  \\
\textit{\textbf{Continuing from before proof...}}\\
This is a contradiction $\therefore $ we can say (unconditionally) that TWO CLIQUES is not $\textbf{PSPACE}$-complete $\square $
\end{enumerate}
\section{Algorithmic Game Theory} 
\item \textbf{Question 3}
\begin{enumerate}
\item Find all the pure strategy equilibria. \\
There are $n \choose 2 $ pure strategy equilibria in this game; where exactly two people are volunteering. This is the case because in these circumstances if one of the players were to stop volunteering then they would decrease their score from 9 to 8 $\Rightarrow $ this would no longer be an equilibrium. Also, if one of the players who was currently not volunteering changed to volunteering then they would decrease their score from 10 to 9 $\Rightarrow $ this would no longer be an equilibrium.
\item Find all the symmetric mixed strategy equilibria. \\
\textit{\textbf{Note:}} $P(n)$ is the probability function describing the probability of $n$ players volunteering. \par
\textit{Score for volunteering}: 9\\
\textit{Score for not volunteering}:\\ $10(P(\geq 2)) + 8(P(<2)) = 10(P(\geq 2)) + 8(1-P(\geq 2))$ \\
$10P(\geq 2) + 8(1-P(\geq 2)) = 9$\\
$8(1-P(0)-P(1))+10(P(0)+P(1)) = 9$\\
$10P(0)-8P(0)+10P(1)-8P(1)+8 = 9$\\
$2(P(0)+P(1))=1$\\
$P(0)+P(1) = \frac{1}{2}$\par
Now using the binomial distribution (${n \choose r}q^r (1-q)^{n-r}$) we can see that the probability of $P(0)$ and $P(1)$ are as follows:\\
$P(0) = p^{n-1}$, $P(1)= (n-1)(1-p)p^{n-2} $\par
Using these we can create the following polynomial to attempt to solve for $p$ which is the probability of \textbf{not} volunteering:\\
$p^{n-1}+(n-1)(1-p)p^{n-2} -\frac{1}{2} = 0 $\\
$(4-2n)p^{n-1} + 2(n-1)p^{n-2} - 1 = 0$\\
$2(n-2)p^{n-1}+2(1-n)p^{n-2} +1 = 0$\par
\textbf{Intermediate Value Theorem} can now be used to show that a solution to this polynomial exists:\\
$F(p)=2(n-2)p^{n-1}+2(1-n)p^{n-2} +1$\\
$F(0)=1$\\
$F(1)=2(n-2)+2(1-n)+1 =-4+2+1=-1$\\
$\therefore $ $y=F(p)$ is such that $F(1) < F(p) < F(0)$ meaning there is some $c$ between $0<c<1$ for which $c$ is a solution of $F(p)$ and since $F(p)=0$ above must be solution to $F(p)=0$ for our symmetric mixed strategy; our probability $\square$
\end{enumerate}
\item \textbf{Question 4} \\
Show that a Boolean function $f(p_1, p_2)$ can be computed by a three-player game. That is, each player has two pure strategies that are the Boolean values $T$ and $F$, and there’s a pure-strategy Nash equilibrium if and only if the strategy of player three, $p_3 =f(p_1, p_2)$ where $p_1$ and $p_2$ are the strategies of players one and two, respectively.\par

The construction of the game is as follows: \\
We have three players 1,2,and 3. Each player has two pure strategies $T$ and $F$. The players receive a score of 1 if the strategies $p_1,p_2,p_3$ are such that $p_3=f(p_1,p_2)$, and a score of 0 otherwise.\\
Using the \textbf{above} construction we want to prove that there is a pure strategy Nash Equilibrium $\iff $ the strategy of player 3, $p_3$, is such that $p_3=f(p_1,p_2)$.\\
($\Rightarrow $)\\
Pure Strategy Nash Equilibrium (NE) $\Rightarrow $ $p_3=f(p_1,p_2)$\\
Suppose that there exists some pure strategy NE such that $p_3 \neq f(p_1,p_2)$. This would mean that the score for the players would be 0 as such there would be a unilateral change in strategy that results in a higher score $\Rightarrow $ this is not a pure strategy NE. $\therefore $ we have a contradiction which means that Pure Strategy NE $\Rightarrow $ $p_3=f(p_1,p_2)$.\\
($\Leftarrow $)\\
$p_3=f(p_1,p_2)$ $\Rightarrow $ Pure Strategy Nash Equilibrium\\
If $p_3=f(p_1,p_2)$ then the players would get a score of 1 which is the maximum score in the game, this means that any change in strategy would result in a lower or equal score $\Rightarrow $ this is a pure strategy equilibrium. $\therefore $ $p_3=f(p_1,p_2)$ $\Rightarrow $ Pure Strategy NE.\\
As demonstrated we have proven the correspondence of $p_3 =f(p_1, p_2)$ and a pure strategy NE, and in turn we have shown that a Boolean function $f(p_1,p_2)$ can be computed by a three-player game $\square $
\end{enumerate}
\end{document}

